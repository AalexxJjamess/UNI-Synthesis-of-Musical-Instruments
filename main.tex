\documentclass{article}
\usepackage[utf8]{inputenc}

\title{Synthesis}
\author{Alex Booth\\ a.booth9@edu.salford.ac.uk}
\date{October 2022}

\begin{document}

\maketitle
\section{Abstract}
    \textit{Here goes the abstract}
\section{Introduction}
Musical instruments have been part of human culture and craft since prehistory, playing a part in both written and verbal art, ceremony and celebration \cite{rault}.
This report describes an attempted recreation using digital synthesis of two acoustic musical instruments from the western European musical tradition: A Mandolin and a flute.
Acoustic musical instruments use an excited physical component to generate waves which are manipulated and shaped by the body or construction of the musical instrument. % CITE
The generation of these waves, and their manipulation by the body of the musical instrument can be modelled using a series of oscillators, filters and modulators.
Audio synthesis using analog circuitry was explored as soon as simple oscillators were readily available. Leading to instruments such as the theremin.  % WHEN?
In the 1970s, FM synthesis was beginning to take form as a method of musical instrument emulation. Chowning's work outlined specific FM techniques for the emulation of various instruments \cite{chowning1973synthesis}.


\section{Theory}
\section{Methodology}
\section{Discussion}
\section{Conclusions}
\section{Appendix}
\subsection{Code}



\bibliographystyle{IEEEtran} % We choose the "plain" reference style
\bibliography{theBib.bib}

\end{document}


